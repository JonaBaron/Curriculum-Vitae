%----------------------------------------------------------------------------------------------------------------
%.     Ceci est le préambule, il n'y a rien à changer dans cette partie du document.
%----------------------------------------------------------------------------------------------------------------

%On définit le type de document, de papier et la grosseur des caractères.
\documentclass[letterpaper,11pt]{article}

%Trois lignes liées à l'ordinateur avec lequel on produit le document.
\usepackage[utf8]{inputenc}
\usepackage[T1]{fontenc}
\usepackage[top=2cm,bottom=2cm,left=2.5cm,right=2.5cm,includehead=true,headheight=1cm]{geometry}%Pour ajustement des marges.
\usepackage{amsbsy}%Accès à des symboles mathématiques
\usepackage{amsfonts}%Accès à des symboles mathématiques
\usepackage[leqno]{amsmath}%Pour que les équations soient numérotées à gauche.
\usepackage{amssymb}%Accès à des symboles mathématiques
\usepackage{bm}%Pour mettre en gras n'importe quel symbole.
\usepackage{ctable}%Pour obtenir des lignes d'épaisseur variable dans les tableaux.
\usepackage{enumerate}%Pour les listes d'énumération.
\usepackage{fancyhdr}%Permet de modifier les entêtes.
\usepackage[bottom]{footmisc}%Pour avoir les notes vraiment en bas de la page.
\usepackage{framed}%Pour pouvoir mettre une boite autour d'un élément.
\usepackage{graphicx}%Pour insérer facilement une figure.
\usepackage{multicol}%Pour créer un document à plusieurs colonnes.
\usepackage{parskip}%Pour pouvoir modifier l'espace entre les paragraphes.
\usepackage{setspace}%Pour pouvoir modifier l'espace entre les lignes.
\usepackage{siunitx}%Pour les nombres et les unités.
\usepackage{tabularx}%Utile avec les tableaux
\usepackage{upgreek}%Pour accéder aux lettres grecques en romain.
\usepackage{xcolor}%Pour accéder à toutes les couleurs dans le document.
\usepackage[b]{esvect}%Permet de créer des vecteurs.
%Pour créer des colonnes de tableau avec package « tabularx »
\newcolumntype{L}[1]{>{\raggedright\arraybackslash}p{#1}}
\newcolumntype{C}[1]{>{\centering\arraybackslash}p{#1}}
\newcolumntype{R}[1]{>{\raggedleft\arraybackslash}p{#1}}

%Pour que les tableaux s'appellent des tableaux.


\sisetup{locale = FR,separate-uncertainty=true,list-units=repeat,multi-part-units=brackets,per-mode=symbol}%Pour que les nombres soient écrits avec une virgule.

%Les quatre lignes qui suivent définissent l'espace entre les paragraphes, l'indentation, l'espace entre les lignes et la distance entre une boite et ce qu'elle contient.
\parskip=10pt 
\parindent=0pt
\onehalfspacing
\fboxsep=9pt

%Pour créer des vecteurs avec une lettre normale ou une lettre grecque
\newcommand{\vecteur}[1]{
\vv{\mathbf{#1}}
}
\newcommand{\vecteurg}[1]{
\vv{\boldsymbol{#1}}
}

%Les lignes qui suivent servent à changer le symbole de la numérotation des sections et des sous sections. Demandez-moi de l'aide si vous en avez besoin...
\renewcommand\thesection{\arabic{section}}
\renewcommand\thesubsection{\thesection.\arabic{subsection}}
\newcolumntype{P}[1]{>{\centering\arraybackslash}p{#1}}
\newcolumntype{M}[1]{>{\centering\arraybackslash}m{#1}}
%Voici les options qui sont possibles :
%\arabic (1, 2, 3, ...)
%\alph (a, b, c, ...)
%\Alph (A, B, C, ...)
%\roman (i, ii, iii, ...)
%\Roman (I, II, III, ...)
%\fnsymbol (?, †, ‡, §, ¶, ...)
%-----------------------------------------------------------------------------------------------------------
%          Ici on insère les informations pour la page titre.
%-----------------------------------------------------------------------------------------------------------
\begin{document}
\begin{center}
% Intro
{\LARGE\textbf{Jonathan Mehmannavaz}}
\\
mehmannavazjonathan@gmail.com \textbar{} +1 (514)-926-7814
\\
LinkedIn: linkedin.com/in/jonathan-mehmannavaz/
\\
GitHub: github.com/JonaBaron
\\
Personal website: JonaBaron.github.io
\end{center}
%-----------------------------------------------------------------------------------------------------------
% SUMMARY section
%-----------------------------------------------------------------------------------------------------------
\textbf{SUMMARY OF SKILLS AND QUALIFICATIONS}\par
\vspace{-20pt}
\rule{\textwidth}{0.4pt}
% Operating system
{\small\textbf{Operating System}} \textbar{} Windows, Mac OS, Linux

%Software
{\small\textbf{Software}} \textbar{} Microsoft 365, Teams, Power BI, Adobe Suite, Visual Paradigm, Zoom, AutoCAD, SolidWorks, Altium Designer, GitHub, Apache web server, LaTeX, ModelSim, Google Cloud, SAP, Xcode

% Programing
{\small\textbf{Programming}} \textbar{} C/C++, Java, Python, Jupyter, JavaScript, HTML, CSS, Swift, SwiftUI, PL-SQL, Node.js, Bash, Assembler, VHDL, PostgreSQL, TensorFlow, Keras, Scikit-learn, Agile (Scrum)

% Hardware
{\small\textbf{Hardware}} \textbar{} Arduino, Allan-Bradley PLC, ARM processor, FPGA board, Oracle VM Virtual Box

% Languages
{\small\textbf{Languages}} \textbar{} French (written and spoken) and English
(written and spoken)

% Certification
{\small\textbf{Licenses \& Certification}} \textbar{} Tensorflow for Deep Learning udemy certification, Power BI udemy certification, iOS developpement udemy cetification

%-----------------------------------------------------------------------------------------------------------
% EDUCATION section
%-----------------------------------------------------------------------------------------------------------
\textbf{EDUCATION}\par
\vspace{-20pt}
\rule{\textwidth}{0.4pt}
%Concordia
\textbf{Bachelor of Engineering - Computer Engineering Co-op \hfill 2022 --2026 (Expected)}
\\Concordia University, Montréal (Québec)

\begin{itemize}
\setlength{\itemsep}{-3pt}
\item
  Member of the Institute for Co-operative Education
\item
  Relevant courses: Object-oriented programming with Digital system
  design.
\end{itemize}

% ETS
\textbf{Technological Academic Path in Engineering \hfill 2021 -- 2022}
\\École de technologie supérieure -- ETS, Montréal (Québec)

\begin{itemize}
\setlength{\itemsep}{-3pt}
\item
  Relevant courses: Computer and Technical drawing for engineers with
  Programmable Logic Controllers (PLCs) and Sequential Logic.
\end{itemize}

% Cégep
\textbf{Diploma of College Studies in Pure Sciences \hfill 2019 -2021}
\\
Cégep régional de Lanaudière à L'Assomption, L'Assomption (Québec)

% Collège
\textbf{Secondary School Diploma (International Baccalaureate (IB)
Program) \hfill 2014 - 2019}
\\
Collège de l'Assomption, L'Assomption (Québec)

%-----------------------------------------------------------------------------------------------------------
% WORK section
%-----------------------------------------------------------------------------------------------------------

\newpage
\textbf{WORK EXPERIENCE}\par
\vspace{-20pt}
\rule{\textwidth}{0.4pt}
\textbf{COOP Supervisor internship \hfill September 2023 --December 2023}
\par
\vspace{-15pt}
% Pratt
Pratt \& Whitney Canada, St-Hubert (Québec)

\begin{itemize}
\setlength{\itemsep}{-3pt}
\item
  Managed employees in a unionized environment by working closely with
  the supervisor (ensure smooth day-to-day operations, assign tasks, act
  as a resource person for the department).
\item
  Acted as a leader to execute the change management project that touched the department while reducing non-quality. 
\item
Leaded strategic initiatives to enhance productivity through KPI monitoring and implementation of RF scanning systems, resulting in continuous departmental improvements.
\end{itemize}

Core Competencies: Communication and presentation skill, autonomy,
leadership, sense of priorities, initiative and diplomacy
\vspace{-5pt}
% Réno-Dépôt

\textbf{Seasonal Sales Advisor \hfill May 2022 - August 2022}
\par
\vspace{-15pt}
Réno-Dépôt, Charlemagne (Québec)

\begin{itemize}
\setlength{\itemsep}{-3pt}
\item 
Consistently exceeded sales targets while maintaining high customer satisfaction through efficient problem-solving, effective inventory management, and strong organizational skills.
\end{itemize}

Core Competencies: Oral Communication, Teamwork
\vspace{-5pt}

% Staples
\textbf{Computer Associate} \hfill \textbf{June 2021 - August 2021} 
\par
\vspace{-15pt}
Bureau en Gros(Staples), Joliette (Québec)

\begin{itemize}
\setlength{\itemsep}{-3pt}
\item
Used technical expertise to drive top computer sales in Montreal through comprehensive customer support, including setup, phone assistance and initial contact to purchase completion.
\item
Demonstrated leadership capabilities by successfully managing team operations and sales targets during management absence while maintaining high service standards.
\end{itemize}
Core Competencies: Oral Communication, Adaptability, Teamwork, Attention
to Detail
\vspace{-5pt}

%-----------------------------------------------------------------------------------------------------------
% Project section
%-----------------------------------------------------------------------------------------------------------
\textbf{PROJECTS}\par
\vspace{-20pt}
\rule{\textwidth}{0.4pt}

% AI
\textbf{TensorFlow for Deep Learning Bootcamp Projects \hfill December - January 2025}
\begin{itemize}
\setlength{\itemsep}{-3pt}
\item
Participated in a Udemy course to build multiple models using TensorFlow, Keras and Scikit-learn on Python using Google Colab platform.
\item
Built neural network for regression, classification, computer vision, natural language processing and time series projections.
\end{itemize}
Core Competencies: Self-taught, Data Science, Problem-solving Mindset, ML/AI Developments

% Hovercraft
\textbf{Introductory Engineering Team Design Project \hfill September - December 2024}
\begin{itemize}
\setlength{\itemsep}{-3pt}
\item
Leaded a autonomous hovercraft development with 5-person team, utilizing Arduino, Blender, SolidWorks Fluid Simulation, Matlab and embedded programming.
\item
Managed full project lifecycle from design to competition and achieving successful autonomous navigation demonstration.
\end{itemize}
Core Competencies: Autonomous Systems Design, Embedded Code Design, Team Leadership, Project Management, Technical Documentation
Competencies

% Web app
\textbf{Software Process and Practices Project \hfill September - December 2024}

\begin{itemize}
\setlength{\itemsep}{-3pt}
\item
Leaded development of a comprehensive peer assessment platform with 7-person team, utilizing Node.js,
Express.js, and PostgreSQL for backend architecture and EJS with Bootstrap for frontend design.
\item
Managed agile development process through 4 sprints, coordinating cross-functional team tasks
and delivering incremental improvements based on stakeholder feedback.
\end{itemize}
Core Competencies: Full-Stack Development, Agile Project Management, Web App Design, Cloud Services

%-----------------------------------------------------------------------------------------------------------
% VOLUNTEER section
%-----------------------------------------------------------------------------------------------------------
\textbf{VOLUNTEER WORK AND EXTRA-CURRICULAR ACTIVITIES}\par
\vspace{-20pt}
\rule{\textwidth}{0.4pt}
\textbf{International Solidarity Trip to Guatemala} \hfill \textbf{April 2019}

\begin{itemize}
\setlength{\itemsep}{-3pt}
\item
  Participated in community work and activities such as building
  concrete floors, working with underprivileged youth, and setting up
  games for Guatemalan seniors.
\end{itemize}

\textbf{Youth Committee for Amnesty International \hfill 2016-2017-2018}

\begin{itemize}
\setlength{\itemsep}{-3pt}
\item
  Participated in Amnesty International events and info booths at
  Collège de l'Assomption, supporting fundraising, awareness, and
  advocacy for human rights.
\end{itemize}

\textbf{Knights of Columbus \hfill 2015-2016}

\begin{itemize}
\setlength{\itemsep}{-3pt}
\item
  Contributed to community events by volunteering at the Knights of Columbus in St-Sulpice.
\end{itemize}

%-----------------------------------------------------------------------------------------------------------
% INTERESTS section
%-----------------------------------------------------------------------------------------------------------

\textbf{INTERESTS}\par
\vspace{-20pt}
\rule{\textwidth}{0.4pt}
\textbf{Football Team "Les Sphinx" \hfill 2017, 2018, 2019}

\begin{itemize}
\item
  Regional RSEQ gold and silver medal winner.
\end{itemize}

\textbf{Basketball Team "Les Sphinx" \hfill 2015-2016}

\end{document}
